@article{huzaifah-features,
  author    = {Muhammad Huzaifah},
  title     = {Comparison of Time-Frequency Representations for Environmental Sound
               Classification using Convolutional Neural Networks},
  journal   = {CoRR},
  volume    = {abs/1706.07156},
  year      = {2017},
  url       = {http://arxiv.org/abs/1706.07156},
  archivePrefix = {arXiv},
  eprint    = {1706.07156},
  timestamp = {Sat, 02 Feb 2019 16:31:57 +0100},
  biburl    = {https://dblp.org/rec/bib/journals/corr/Huzaifah17},
  bibsource = {dblp computer science bibliography, https://dblp.org}
}


@INPROCEEDINGS{chachada-survey,
author={S. {Chachada} and C. -. J. {Kuo}},
booktitle={2013 Asia-Pacific Signal and Information Processing Association Annual Summit and Conference},
title={Environmental sound recognition: A survey},
year={2013},
volume={},
number={},
pages={1-9},
keywords={acoustic signal processing;audio signal processing;feature extraction;environmental sound recognition;audio recognition;speech signals;music signals;nonstationary characteristics;signal temporal characteristics;signal spectral characteristics;sequential learning methods;long-term environmental sound variation;environmental sound processing schemes;nonstationary ESR techniques;Mel frequency cepstral coefficient;Feature extraction;Hidden Markov models;Speech;Accuracy;Atomic clocks},
doi={10.1109/APSIPA.2013.6694338},
ISSN={},
month={Oct},}


# https://towardsdatascience.com/dont-use-dropout-in-convolutional-networks-81486c823c16
